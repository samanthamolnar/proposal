\addcontentsline{toc}{chapter}{Abstract}

\begin{abstract}

Power systems today are undergoing extreme changes with the integration of renewable generation sources such as solar and wind. With these new technologies come new challenges. Traditional generation sources, such as coal and nuclear, provide inertial support, which helps maintain system stability when an outage occurs. However, solar and wind generation do not, on their own, provide this support, which potentially makes the system more vulnerable to failures. The effect on stability as power engineers introduce more renewables into the generation mix remains an open question. 

One way some have proposed to address this problem is by engineering a way for renewable generation sources to provide inertial support. Inverters, which convert DC power to AC power for renewable generation, are outfitted with control algorithms that allow wind and solar to provide inertial support synthetically. However, this still does not answer the question of how much inertial support these generation sources need to provide, and what happens if they are unable to do so.   This is the motivating question for my research. 

I am specifically interested in studying how the inertial makeup of a power system affects the severity of cascading failures, as well as understanding the dynamic interactions that occur during the cascade.  In particular, I will test how total system inertia 


I will use a model developed by Adilson Motter, that defines a continuous time phase-space definition of the power system dynamics, and shows that this system can be defined by a Hamiltonian-like equation.  The energy function associated with this Hamiltonian quantifies the stability of the system, because it remains unchanged when the generators are synchronized and there are no failures.  [What else do I need here?]  Say what the shortcomings of the energy function are, as a metric for what you want to measure.  Also, you should probably give a quick summary of network/motif-based research on power-grid failures, since you’re going to build on that later.  A good place for that might be after your “previous research …” sentence above, since that’s a way that people have quantified risk, right?

I plan on studying how power system inertia influences whether a cascade occurs, how severe it is, and which elements fail.  I wish to answer the specific question: given the same contingency, will the same cascade result for different inertial values?  (The previous paragraph outlined broader goals.  You should make sure they line up: either trim things down above or add specific questions here.)  In principle, the stable points of a power system are not influenced by the inertia of the generators, but different overloads can occur due to different inertial values, which does influence which stable points are reachable.  I will run experiments for this study using two models: the continuous model mentioned previously, and a state-of-the-art PSLF model of the Western Interconnection under various renewable energy scenarios.  The first model specifically quantifies the inertia values of generators and an energy functions, while the second is a more realistic simulation of a system.  That sentence is a bit confusing.  It needs to answer the reader’s question “why these two models  — why not just Adilson’s or just the PSLF one?  And why not others as well as these two…why are these sufficient?” Does the PSLF model incorporate inertia, for example?  If it doesn’t, why is it a useful part of your work?  If it does, why use Adilson’s?
Somewhere here you need to talk about your test case: which grid?  Why?  How are you going to change the inertia and why is that realistic?

%%
I will assess the effect of inertia on cascading failure in two ways: the first is by calculating the distribution of cascade size and its probability (explain what you mean there); the second is by analyzing the differences in the results of the same contingency with different inertial values.  The latter is important to understand because even if the risk of large cascades does not change, knowing which contingencies lead to large cascades will alter how operators need to plan day-to-day operations and infrastructure updates.  You need a sentence here explaining what “distribution of cascade size and probability” means  and saying how you’ll approach that.  In order to understand the differences in particular contingency scenarios I will look at (and do what with which parts of that information in order to answer your specific research question?) which lines actually failed, as well as the interaction among the components in the system, in particular the generators.  What calculations will you do, on what quantities, to study the interactions?  Additionally, I would like to explore a motif-driven analysis of the final steady-states of contingencies to determine if certain substructures occur more frequently, which may be a necessary condition for stability.
[Where do I go from here?]
You need a paragraph saying what’s new here: that you’re using an existing model in a new way to explore a new research question, applying (motif) and updating existing metrics (energy function) in ways that will be needed to help answer that question.  How will you know that you’re done?



\end{abstract}